% This must be in the first 5 lines to tell arXiv to use pdfLaTeX, which is strongly recommended.
\pdfoutput=1
% In particular, the hyperref package requires pdfLaTeX in order to break URLs across lines.

\documentclass[11pt]{article}

% Remove the "review" option to generate the final version.
\usepackage{acl}

% Standard package includes
\usepackage{times}
\usepackage{latexsym}

% For proper rendering and hyphenation of words containing Latin characters (including in bib files)
\usepackage[T1]{fontenc}
% For Vietnamese characters
% \usepackage[T5]{fontenc}
% See https://www.latex-project.org/help/documentation/encguide.pdf for other character sets

% This assumes your files are encoded as UTF8
\usepackage[utf8]{inputenc}

% This is not strictly necessary, and may be commented out,
% but it will improve the layout of the manuscript,
% and will typically save some space.
\usepackage{microtype}

% If the title and author information does not fit in the area allocated, uncomment the following
%
%\setlength\titlebox{<dim>}
%
% and set <dim> to something 5cm or larger.

\title{Testing Egunean Behin Visual Question Answering Dataset with BLIP}

% Author information can be set in various styles:
% For several authors from the same institution:
% \author{Author 1 \and ... \and Author n \\
%         Address line \\ ... \\ Address line}
% if the names do not fit well on one line use
%         Author 1 \\ {\bf Author 2} \\ ... \\ {\bf Author n} \\
% For authors from different institutions:
% \author{Author 1 \\ Address line \\  ... \\ Address line
%         \And  ... \And
%         Author n \\ Address line \\ ... \\ Address line}
% To start a seperate ``row'' of authors use \AND, as in
% \author{Author 1 \\ Address line \\  ... \\ Address line
%         \AND
%         Author 2 \\ Address line \\ ... \\ Address line \And
%         Author 3 \\ Address line \\ ... \\ Address line}

\author{Julen Etxaniz \\
  University of the Basque Country (UPV/EHU) \\
  \texttt{jetxaniz007@ikasle.ehu.eus}}

\begin{document}
\maketitle
\begin{abstract}
Egunean Behin is a popular Basque quiz game. The game consists on answering 10 daily multiple choice questions.
Questions were translated to English because VQA models like BLIP are mainly trained on English questions.
Three types of questions from the game were selected: figures, cubes and maze. All the images and questions were generated automatically.
There are multiple questions for each image. Questions require counting figure, colors, cubes and understanding the dimensions of the pictures.
Each question has one correct and two wrong answers. These can be used for multiple choice question answering.
\end{abstract}

\section{Introduction}

Egunean Behin is a popular Basque quiz game. The game consists on answering 10 daily multiple choice questions.
Questions were translated to English because VQA models like BLIP \cite{li2022blip} are mainly trained on English questions.
Three types of questions from the game were selected: figures, cubes and maze. All the images and questions were generated automatically.
There are multiple questions for each image. Questions require counting figure, colors, cubes and understanding the dimensions of the pictures.
Each question has one correct and two wrong answers. These can be used for multiple choice question answering.

\section{Related Work}

\section{Material and Methods}

\section{Results}

\section{Conclusions}

\bibliography{egunean_behin_vqa}
\bibliographystyle{acl_natbib}

\end{document}
